
\newgidx{glosario}{Glosario}
\DTLgidxSetDefaultDB{glosario}
%centroide
%\newterm[description={}]{}
\newterm[description={Conexión funcional entre dos sistemas, programas, dispositivos o componentes de cualquier tipo},plural={interfaces}]{interfaz}
\newterm[description={Entorno de escenas u objetos de apariencia real, generado mediante tecnología informática}]{realidad virtual}
\newterm[description={Objeto complejo cuyos componentes se relacionan con al menos algún otro componente.}]{sistema}
\newterm[description={Hace referencia a las transformaciones conformes, en los cuales se mantienen los ángulos.}]{conforme}


\newterm[description={Del inglés "picture element" es la unidad básica de una imagen}, plural={pixeles}]{pixel}

\newterm[description={Es una figura geométrica que define una ubicación especifica del espacio.},plural={puntos}]{punto}

\newterm[description={Conjunto de puntos, habitualmente describen superficies. }, plural={nubes de puntos}]{nube de puntos}


\newterm[description={Elemento geométrico de dos dimensiones, que contiene infinitos puntos y rectas.},plural={planos}]{plano}
\newterm[description={elemento geométrico descrito por una superficie curva cuyos puntos equidistan al centro de esta.},plural={esferas}]{esfera}
\newterm[description={Elemento geométrico descrito por la revolución de un rectángulo.},plural={cilindros}]{cilindro}

\newterm[description={Representación visual de algún objeto}]{imagen}

%\newterm[description={}]{vector}

%\newterm[description={espacio vectorial}]{álgebra}

\newterm[description={Dispositivo eléctrico con la capacidad de manifestar magnitudes en variables eléctricas},plural={sensores}]{sensor}

\newterm[description={Subrutina perteneciente a la clase de algún programa para realizar una acción}, ]{método}

\newterm[description={Conjunto de instrucciones ordenadas y finitas para realizar alguna actividad}]{algoritmo}
\newterm[description={Programa que permite al sistema operativo interaccionar con un periférico}]{controlador} 
\newterm[description={Programa que remueve parte de la información dada. }]{filtro} 


%ToF
%AG
%agc
%ae
%RANSAC
%PCL
