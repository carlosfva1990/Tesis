
\chapter{Resumen}
En este trabajo se presenta una propuesta para el desarrollo de una interfaz que permita la interacción de objetos reales con un mundo virtual, usando un clasificador objetos son representados por geometrías simples. Modificando el proceso de clasificación usando álgebra geométrica conforme (AGC) se plantea mejorar el proceso.


Con el sensor Kinect se obtienen las nubes de puntos de los objetos, se realiza la estimación de parámetros para los modelos de la esfera, el plano y el cilindro, usando el método RANSAC, y posteriormente seleccionar el modelo que mejor represente al objeto.

El sistema se compara contra el mismo pero en lugar de usar AGC utiliza métodos RANSAC de la biblioteca Point Cloud Library (PCL). Los resultados muestran valores similares entre los sistemas así como algunas ventajas individuales para cada sistema.
 

\chapter{Abstract}
It is presented the development of an interface between real objects and virtual reality, with the help of an object classifier the objects are represented by simple geometric forms.  By modifying the method using conformal geometric algebra (CGA) is expected to improve the classifier.

A parameter estimation is made for the sphere, plane, and cylinder using the RANSAC estimator and the Kinect to get the point clouds.

The system is compared with a modified version of itself which use the RANSAC methods from the Point Cloud Library (PCL) instead of the CGS methods, the results show some similarity between systems and some benefits for each system.

