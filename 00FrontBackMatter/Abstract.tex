
\chapter{Resumen}
En este trabajo se plantea medir el proceso usado para clasificar objetos fisicos que interactuan en el mundo virtual. Usando un clasificador objetos se genera la representacionde estos usando geometrías simples, luego de modificar el proceso usando álgebra geométrica conforme (AGC) se plantea mejorar el proceso.


Se utiliza el sensor Kinect para obtener un a representacion virtual de los objetos fisicos de los cuales se obtienen las nubes de puntos, usando el metodo RANSAC se realiza la estimación de parámetros para los modelos geometricos de la esfera, el plano y el cilindro, y seleccionar el modelo geometrico que coincida en mayor medida con la nuve de puntos del objeto.

El sistema se compara contra el mismo pero en lugar de usar la biblioteca Point Cloud Library (PCL) y los modelos euclidianos de las geometrias se utilizan los modelos geometricos representados con AGC. Los resultados muestran valores similares entre los sistemas así como algunas ventajas individuales para cada uno.
 

\chapter{Abstract}
It is presented the development of an interface between real objects and virtual reality, with the help of an object classifier the objects are represented by simple geometric forms.  By modifying the method using conformal geometric algebra (CGA) is expected to improve the classifier.

A parameter estimation is made for the sphere, plane, and cylinder using the RANSAC estimator and the Kinect to get the point clouds.

The system is compared with a modified version of itself which use the RANSAC methods from the Point Cloud Library (PCL) instead of the CGS methods, the results show some similarity between systems and some benefits for each system.

