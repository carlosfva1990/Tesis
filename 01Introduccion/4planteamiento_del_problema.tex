
\section{Planteamiento del Problema}

    En la actualidad existen diversos sistemas que permiten la interacción y el reconocimiento de objetos del mundo real, pero la mayoría de ellos se limitan a trabajar solo con objetos o entornos ya establecidos. Sistemas de realidad virtual como HTC Vive \cite{VIVEDis84:online} y Oculus Rift \cite{OculusRi96:online} interactúan con objetos virtuales usando un mando especial, pero no detectan objetos con los cuales el usuario pudiera colisionar, En sistemas de realidad mixta como los Holo-Lens de Microsoft \cite{HoloLens} o en visión por computadora  se puede obtener la posición de las paredes de una habitación y encontrar obstáculos, pero la interacción del usuario es limitada. Por lo anterior, se hace clara la falta de proyectos que permitan la interacción con objetos cotidianos  sin la necesidad de conocer a detalle todas las características del objeto, solo interesando algunas de ellas, tales como su tamaño, posición y geometría.
    