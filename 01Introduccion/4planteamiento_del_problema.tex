
\section{Planteamiento del Problema}

    En la actualidad existen diversos sistemas que permiten la interacción y el reconocimiento de objetos del mundo real, pero la mayoría de ellos se limitan a trabajar solo con el usuario, asi como, Sistemas de realidad virtual como HTC Vive \cite{VIVEDis84:online} y Oculus Rift \cite{OculusRi96:online} interactúan con objetos virtuales usando un mando especial, pero no detectan objetos reales con los cuales el usuario pudiera interactuar, En sistemas de realidad mixta como los HoloLens de Microsoft \cite{HoloLens} o en visión por computadora se puede obtener la posición de las paredes de una habitación y encontrar obstáculos, pero la interacción del usuario es limitada.\\
    
    Por lo anterior, se hace clara la falta de proyectos que permitan la interacción con objetos cotidianos, como los proyectos que usan tecnicas con puntos caracteristicos, los cuales tienen el inconveniente de que es necesario realizar un escaneo previo de cada objeto a utilizar, lo cual genera la necesidad de un usuario con conocimientos especializados para su uso, esto es util cuando la interaccion entre el objeto y el usuario es el foco del sistema, pero al buscar una interaccion ya que se busca una interaccion simple con el ambiente, desaparece la necesidad de conocer a detalle todas las características del objeto, solo interesando algunas de ellas, tales como su tamaño, posición, rotacion y geometría general.
     
