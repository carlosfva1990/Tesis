%\section{Motivación}
Actualmente los sistemas de \gls{realidad virtual} se encuentran en auge tanto en el ámbito comercial como en el desarrollo tecnológico, esto gracias al incremento en la capacidad de cómputo y la reducción del tamaño de algunos dispositivos electrónicos. Tecnologías como Google Cardboard, HTV VR Headset, Microsoft HoloLens, etc. han sido tecnologías pioneras permitiendo al usuario experimentar la realidad virtual de forma más personal e intuitiva.\\

De manera similar a cuando la tecnología de las pantallas táctiles llegó a los celulares y forzaron la creación de nuevas interfaces para la interacción entre el usuario y la computadora, las nuevas tecnologías de realidad virtual se enfrentan al desarrollo de nuevas \glspl{interfaz}, se entiende como interfaz a un programa que actúa de intermediario entre las señales de entrada proporcionadas por el usuario las cuales filtra y ajusta para crear información que pueda utilizar el sistema.\\

Estas nuevas interfaces son particularmente complicadas ya que se busca una interacción natural e intuitiva para el usuario y su entorno, y no solo la interacción entre usuario y el mundo virtual. Algunas empresas como Mircosoft con el proyecto de HoloLens \cite{HoloLens} apuntan su desarrollo al conjunto de aplicaciones que se encuentran entre la realidad virtual y la realidad aumentada (realidad mixta), esto conlleva una interacción entre elementos reales, elementos virtuales y el usuario o usuarios.\\

Este trabajo esta motivado por la necesidad de la creación de interfaces que permitan la interacción entre el usuario y el mundo virtual, para esto se busca la representación de objetos reales basada en su geometría.\\

Se presenta la propuesta de un sistema para la representación de objetos y la comparación del sistema desarrollado en dos marcos de trabajo matemáticos distintos que aspira a reducir el tiempo de computo necesario para la representación de estos objetos.\\

El problema de investigación es consecuencia del desarrollo de las nuevas tecnologías y la necesidad de integración  conocimiento para la mejora tecnológica.\\

En efecto, el problema se centra en generar una interfaz que le permita integrar objetos reales a un ambiente virtual así como comparar su despeño modificando el paradigma matemático, usando el Álgebra Geométrica Conforme (AGC).\\

Para este trabajo establecen las siguientes faces de desarrollo:

\begin{enumerate}
	\item Análisis preliminar,
	\item Desarrolló del sistema,
	\item Modificación del paradigma matemático(usando AGC),
	\item Comparación de los sistemas
\end{enumerate}







