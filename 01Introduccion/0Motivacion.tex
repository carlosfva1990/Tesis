%\section{Motivación}
edit 1 Actualmente los sistemas de \gls{realidad virtual} se encuentran en auge tanto en el ámbito comercial como en el desarrollo tecnológico, esto gracias al incremento en la capacidad de cómputo y la reducción del tamaño de algunos dispositivos electrónicos. Tecnologías como Google Cardboard, HTV VR Headset, Microsoft Hololens, etc. han sido tecnologías pioneras permitiendo al usuario experimentar la realidad virtual de forma más personal e intuitiva.\\

De manera similar a cuando la tecnología de las pantallas táctiles llegó a los celulares y forzaron la creación de nuevas interfaces para la interacción entre el usuario y la computadora, las nuevas tecnologías de realidad virtual se enfrentan al desarrollo de nuevas \glspl{interfaz}.\\

Estas nuevas interfaces son particularmente complicadas ya que se busca una interacción intuitiva para el usuario, y que no solo se busca la interacción entre usuario y el mundo virtual, algunas empresas apuntan a la intersección entre la realidad virtual y la realidad aumentada (realidad mixta), esto conlleva una interacción entre elementos reales, elementos virtuales y el usuario o usuarios.\\

El problema de interacción entre el usuario y el mundo virtual se ha intentado trabajar usando sistemas de captura de movimiento como los de la empresa VICON \cite{vicon}, usando cámaras infrarrojas capaces de medir la distancia entre la cámara y un reflector de luz infrarroja es posible calcular la posición y orientación de un objeto. El problema es que estos sistemas son delicados y costosos.\\

La búsqueda de diferentes formas de realizar la interacción entre un entorno virtual, el mundo real y el usuario representa un reto importante.\\
%el cual puede ser desglosado en investigaciones más simples.\\



